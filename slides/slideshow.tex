\documentclass[aspectratio=169, 12pt]{beamer}
 
\usepackage[utf8]{inputenc}
\usepackage{natbib, url}
\usepackage{enumerate, amsmath, amssymb, amsthm}
\usepackage{ragged2e} % make it justified
\justifying

% \usepackage{enumitem}
% \setlist{itemsep = 0pt, topsep = 1pt, leftmargin = 0.6mm}

\usepackage{hyperref}
\hypersetup{colorlinks, citecolor=blue, urlcolor=blue}
\usepackage{booktabs}
\usepackage{graphicx}


%\mode<presentation>{}
%\usepackage{beamerthemesplit} 

\setbeamertemplate{footline}[frame number]
%\setbeamertemplate{headline}{}
 
 
%Information to be included in the title page:
\title{Common Mistakes in Statistical and Methodological Practices in Sport Management Research}

\author{Tonia Mei}

\date{October 21, 2024}

\begin{document}

\frame{\titlepage}


\begin{frame}{Introduction}
  \begin{itemize}
    \item Sport Management Research
    \begin{itemize}
        \item better understand and enhance the governance, marketing, finance, and administration of sports organizations and events
        \item encompasses a broad range of topics (leadership, consumer behavior, sponsorship policy, and organizational effectiveness)
        \item evidence-based decisions are supported by high-quality research
        \item high demand for rigorous academic research to address particular issues posed by the growing sports industry
    \end{itemize}
  \end{itemize}
\end{frame}


\begin{frame}{Introduction}
  \begin{itemize}
    \item Concerns in Research Quality
    \begin{itemize}
        \item despite the significance of sport management research, there are rising concerns of statistical and methodological procedures
        \item typical errors such as misuse of statistical tests and multiple comparisons without replacement, inadequate reporting of effect sizes, misinterpretation of p-values, and an over-reliance on the null hypothesis significance testing will endanger the reliability of the findings
        \item to guarantee that the discipline generates reliable and significant knowledge, these problems underscore the necessity of increased awareness and adoption of best practices in research design, data processing, and reporting
    \end{itemize}
  \end{itemize}
\end{frame}


\begin{frame}{Purpose in the Study}
  \begin{itemize}
      \item Identification
      \begin{itemize}
          \item misuse of statistical tests: researchers frequently utilize statistical tests that are inappropriate for the type of data they are studying
          \item misuse of multiple comparisons without replacement: conducting multiple comparisons without proper correction increases the risk of Type I errors, leading to false positives
          \item inadequate sample sizes: many studies have insufficient sample sizes, leading to a lack of statistical power
          \item inadequate reporting of effect sizes: the practical significance of the data is limited because many studies do not give effect sizes
          \item misinterpretation of results: confusing correlation with causation and an over-reliance on p-values
          \item over-reliance on the null hypothesis: concentration on disproving the null hypothesis while ignoring the results' usefulness
      \end{itemize}
  \end{itemize}
\end{frame}


\begin{frame}{Purpose in the Study}
  \begin{itemize}
      \item Importance of Addressing These Issues
      \begin{itemize}
          \item improved research reliability: researchers can ensure their investigations yield more accurate and trustworthy results by fixing these errors. this keeps false conclusions from influencing the development of theories and real-world applications in sport management
          \item increased validity: appropriate statistical analysis and sound methodological procedures support the internal and external validity of studies, guaranteeing that the conclusions are accurate based on the data and applicable in different settings
          \item enhancing research credibility: resolving these problems makes the field more credible by encouraging more thorough and reliable research, which is essential for the development of sport management as a scientific subject
      \end{itemize}
  \end{itemize}
\end{frame}


\begin{frame}{Research Objectives}
  \begin{itemize}
      \item the primary objective is to direct researchers by identifying common mistakes and recommending best practices
      \item the lack of standardized statistical procedures in sport management research is one of the needs for this study
      \item raise overall quality and credibility of sport management after addressing methodological flaws by promoting it
      \item understand the difference between practical and statistical significance
  \end{itemize}
\end{frame}


\begin{frame}{Frequent Statistical and Methodological Errors}
  \begin{itemize}
      \item Misuse of Statistical Tests
        \begin{itemize}
            \item invalid use of statistical tests that are not suited for the data
            \item using parametric tests prior to fulfilling the required presumptions
        \end{itemize}
      \item Misinterpreting Statistical Results
        \begin{itemize}
            \item confusing the concept of 'correlation' with 'causation'
            \item rejecting/fail to reject the null hypothesis when there is insufficient evidence to support a conclusion
        \end{itemize}
      \item Importance of Using Correct Statistical Tests
        \begin{itemize}
            \item fit the data to the best statistical results
        \end{itemize}
  \end{itemize}
\end{frame}


\begin{frame}{Frequent Statistical and Methodological Errors}
  \begin{itemize}
      \item Sample Sizes
        \begin{itemize}
            \item various research falls short of adequate sample sizes, resulting in statistical tests that are under powered
            \item results may lack generalized and statistical significance
        \end{itemize}
      \item Importance of Adequate Sample Sizes
        \begin{itemize}
            \item explain how to perform power analysis to calculate the necessary sample size
            \item explain how the sample size affects statistical significance and how it affects findings
        \end{itemize}
      \item Over-reliance on the Null Hypothesis
        \begin{itemize}
            \item rejecting/failing to reject the null hypothesis without taking into account its practical implications
            \item p-values are overemphasized, they are the main and only measure of the success of a study
        \end{itemize}
  \end{itemize}
\end{frame}


\begin{frame}{Future Studies for Sport Management Researchers}
  \begin{itemize}
      \item Limitations of the Study
        \begin{itemize}
            \item qualitative nature is based of examples rather than on quantitative assessment of literature
            \item certain suggestions may not be appropriately applied in every study setting, but is something to take a closer look at
        \end{itemize}
      \item Recommendations for Future Practices
        \begin{itemize}
            \item improve research quality by figuring out how findings may boost the rigor of sport management research right away
            \item adopt best practices by encouraging the application of recommended changes in continuing studies
        \end{itemize}
  \end{itemize}
\end{frame}


\begin{frame}{Practical Implications for Sport Management Researchers}
  \begin{itemize}
      \item enhance research quality by determining how results can immediately improve sport management research's rigor
      \item embrace best practices by motivating the use of suggested modifications in ongoing research
      \item encourage data openness by making sure that all statistical procedures and judgments are thoroughly recorded and disclosed
      \item increase the generalizability of your research by using sufficient sample sizes
      \item improve decision-making by using practical and statistical significance in addition to p-values
  \end{itemize}
\end{frame}


\begin{frame}{Future Directions}
  \begin{itemize}
      \item examine the frequency of statistical errors in various journals using a systematic review or meta-analysis
      \item create training initiatives by designing seminars or classes to help researchers advance their statistical knowledge
      \item create a statistical criteria of goals before starting an experiment
      \item promote cooperation between statisticians and sport management researchers
      \item maybe use sophisticated statistical techniques to better handle complex data structures
  \end{itemize}
\end{frame}


\begin{frame}{Conclusion}
  \begin{itemize}
      \item the statistical and methodological rigor in sport management research must be improved
      \item it is important to follow the rules of statistics in order to prevent future common mistakes in the field of sport management research in physical education and exercise science
      \item it takes ongoing training in appropriate statistical methods to lower errors and improve the rigor of research
      \item correcting the common mistakes will increase the reliability of sport management study findings
  \end{itemize}
\end{frame}


\end{document}
