\documentclass[12pt]{article}

%% preamble: Keep it clean; only include those you need

% if the below packages cannot be installed automatically, you can 
% download the required .sty files from CTAN and place them in the
% same location as the .tex file (or upload to overleaf in same
% location (folder) in overleaf

\usepackage{amsmath}
\usepackage[margin = 1in]{geometry}
\usepackage{graphicx}
\usepackage{booktabs}
\usepackage{natbib}
\usepackage{setspace} % for doublespacing
\doublespacing

% highlighting hyper links
\usepackage[colorlinks=true, citecolor=blue]{hyperref}


%% meta data

\title{Proposal: Common Mistakes in Statistical and Methodological Practices of Sport Management Research}
\author{Tonia Mei\\
  Department of Statistics\\
  University of Connecticut
}
\date{September 23, 2024}

\begin{document}
\maketitle


\paragraph{Introduction}
Sport management is crucial to society and plays a part in not only the economics but many more. This type of research extends to a variety of aspects within the sport industry such as the social and cultural aspects where fostering connections are possible in promoting health and performance. If done correctly, the future of entry-level organizations to professional-level organizations may prosper from the effects. As sport management is an industry that constantly expands, the research methods being to impact real-world applications and may be applied to sports policy and practices, which is why the findings of the data need to be valid and able to be replicated and applied.

An article that was performed by Kim and Lee (2019) addresses and analyzes statistical and methodological mistakes errors that arise in sport management research. These two scholars look at common mistakes made in data analysis, research design, and reporting procedures. By providing recommendations of optimal practices and offering advice on how to avoid these mistakes, this article aims to enhance research quality in sport management and seeks to improve the quality and reliability of sport management research across the globe. While addressing the mistakes, they also shed light on the frequent ones in order to provide helpful tips to avoid them. Most common flaws occur across various statistical techniques such as in, regression models, ANOVA model, and factor analysis and are then categorized into pre-, peri-, and post-analysis mistakes.


\paragraph{Specific Aims}
Kim and Lee (2019) aim to identify common statistical and methodological mistakes in sport management research, such as improper use of statistical tests, insufficient sample sizes, and misinterpretation of results. They assess how these errors impact the accuracy, reliability, and validity of research findings, potentially skewing the results. Additionally, the authors provide practical recommendations to help researchers avoid these frequent mistakes and suggest strategies to improve the quality of future studies. By raising awareness of the need for more strict guidelines, the paper seeks to enhance the validity and impact of sport management research.


\paragraph{Data Description}
This data was gathered by the authors, Kim and Lee, by a thorough review of various existing sport management research published articles in sport management journals published from 2001 to 2017. This data will help to identify and categorize common and methodological errors in sport management research. We are able to pinpoint areas where the mistakes stem from and able to provide guidance in areas that require more precise attention. On the basis of quantitative research methods, it is statistically proven that the techniques that are associated with the most common mistakes in sport management research include; regression analysis, ANOVA, and factor analysis. The data that we are going to be identifying are the different categories; pre-, peri-, and post- analysis mistakes. Pre-analysis are errors that occur during the design phase of research, the peri-analysis are mistakes that are made during statistical testing and data analysis, and lastly post-analysis are issues that rise after the analysis of the data.

\paragraph{Research Design and Methods}
Incorrect use of statistical tests, inadequate reporting of effect sizes, misinterpretation of p-values, improper use of multiple comparisons without correction, and an over-reliance on the null hypothesis significance testing are just a few of the problems that come up when looking at common mistakes amongst sport management research. Alternative approaches offered are Bayesian methods or bootstrapping, as they could reduce future mistakes and improve the reliability and validity of sport management outcomes.


\paragraph{Discussion} 
Kim and Lee analyze common statistical and methodological errors in sport management research. Contributing to sport management research will be resolved by preventing future mistakes made, offering specific examples and guidelines are a way to do it. This paper places a spotlight on areas in which statistical and methodological approaches frequently fall short in order to address the need to increase quality of research in sport management. The suggestion of relevant statistical tests, large samples and appropriately interpreting results, have been statistically proven to lower the risk of error.


\bibliography{refs}
\bibliographystyle{asa}

\end{document}