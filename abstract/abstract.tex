\documentclass[a4paper]{article}
%\usepackage{simplemargins}

%\usepackage[square]{natbib}
\usepackage{amsmath}
\usepackage{amsfonts}
\usepackage{amssymb}
\usepackage{graphicx}

\begin{document}
\pagenumbering{gobble}

\Large
 \begin{center}
Abstract: Common Mistakes in Statistical and Methodological Practices of Sport Management Research\\ 

\hspace{10pt}

% Author names and affiliations
\large
 Tonia Mei$^1$, Kim Yukyoum$^2$, J. Lucy Lee$^2$ \\

\hspace{10pt}

\small  
$^1$) Presenter\\
$^2$) Researcher\\

\end{center}

\hspace{10pt}

\normalsize

Sport management research is crucial for shaping both the economic and socio-cultural aspects of the sports industry. Its findings must be accurate, reliable, and applicable to inform sports policy and practices effectively. The authors highlight problems such as p-value misuse and improper test selection, offering solutions like larger samples, better test choices, and alternative methods (e.g., Bayesian or bootstrapping). By addressing these errors, they aim to improve the quality and reliability of sport management research and its practical applications.

Kim and Lee’s analysis identifies issues such as the misuse of p-values, incorrect statistical tests, and inadequate reporting of effect sizes. In response, they offer practical recommendations to improve research quality, including the use of larger samples, proper statistical test selection, and alternative approaches such as Bayesian methods or bootstrapping. By addressing these common errors and offering solutions, the authors aim to enhance the rigor and reliability of sport management research.

\end{document}